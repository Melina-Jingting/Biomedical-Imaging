\subsection{Network Model Performance}
Our comparative results demonstrate that tau propagation in AD can be reasonably well predicted by connectome-based models across the clinical continuum. In particular, the simple Network Diffusion Model (NDM) performed almost as well as a more complex reaction–diffusion model in explaining a canonical tau distribution, highlighting that the brain’s network architecture is a primary determinant of where tau pathology accumulates. In fact, the simplest model NDM captures the relative tau-levels across regions best, with highest r-scores. This finding resonates with the view that brain networks serve as conduits for disease spread, constraining tau to follow connectivity routes. 

\subsection{Amyloid-$\beta$ Influences tau Accumulation}
The significant improvement in performance when incorporating A$\beta$ into the weighted FKPP supports the hypothesis that Amyloid-$\beta$ Influences tau Accumulation. A notable distinction also emerged in the optimal seed regions identified. While both NDM and FKPP converged on the Inferiortemporal region as the optimal seed, the A$\beta$-FKPP model consistently identified the Entorhinal region instead. This finding aligns remarkably well with established neuropathological staging, where the transentorhinal cortex represents the initial site of tau pathology in early Alzheimer's disease progression \citep{dominguez2018three}.

\subsection{Clinically Normal Most Robust Predictions}
Another important result from our study is that model performance varied across individual connectomes, even within the same diagnostic group. While we used group-averaged metrics to compare models, the spread in Pearson r values for individuals (see Table 3) reveals that some participants’ brain networks predicted the tau pattern considerably better than others. The CN group both has better model performance across models and lowest variance in distribution of individual connectome performance. 

\subsection{Limitations}
A key limitation of our study is the use of a group-averaged tau PET map (from 242 individuals) as the ground truth for model fitting, irrespective of cognitive stage. While this provided a uniform comparison metric, it introduced a potential confound. The “tau-positive” PET cohort likely had a distribution of ages and disease stages that do not perfectly align with the CN, MCI, and AD sub-groups whose connectomes we used. Differences in age, amyloid status, vascular health, and other factors between the connectome donors and the tau-PET cohort could influence results. For example, we found that CN connectomes have higher global efficiency and node strength in the hippocampus than AD connectomes. We control for this by evaluatin against null models, but acknowledge that this does not account for all possible confounds.

