Alzheimer’s disease (AD) is a progressive neurodegenerative disorder and the leading cause of dementia worldwide, affecting tens of millions of people and placing an enormous burden on public health \citep{yuHumanConnectomeAlzheimer2021}. Pathologically, AD is characterized by the accumulation of amyloid-$\beta$ (A$\beta$) plaques and tau protein neurofibrillary tangles, which disrupt synaptic function and ultimately drive extensive neuronal loss \citep{goedertAlzheimersParkinsonsDiseases2015, weickenmeierPhysicsbasedModelExplains2019}. Despite concerted research efforts, effective disease-modifying treatments remain elusive, in part owing to the complexity and heterogeneity of AD progression \citep{yuHumanConnectomeAlzheimer2021}. Enhanced mechanistic understanding of how tau pathology spreads, and how A$\beta$ may trigger or accelerate this spread, is therefore crucial for developing earlier diagnostic strategies and more effective interventions. \\

 Tau-propagation models are often based on the \textbf{prion hypothesis}, based on evidence indicates that misfolded proteins such as tau propagate trans-synaptically in a prion-like fashion from vulnerable epicenters to connected brain regions, giving rise to the characteristic Braak staging patterns documented in pathological and PET imaging studies \citep{vogelConnectomebasedModellingNeurodegenerative2023, rajNetworkDiffusionModel2012}. Researchers have thus increasingly turned to \emph{network-based} computational models—relying on connectomics—to better understand how tau spreads via large-scale anatomical or functional pathways in the brain \citep{rajNetworkDiffusionModel2012,toxtobyDataDrivenSequenceChanges2017,weickenmeierPhysicsbasedModelExplains2019,heCoupledmechanismsModellingFramework2023, hompsonCombiningMultimodalConnectivity2024}.\\

 \textbf{The amyloid cascade hypothesis} posits that pathological aggregation of A$\beta$ initiates a cascade culminating in tau-mediated neurodegeneration \citep{goedertAlzheimersParkinsonsDiseases2015}. Indeed, early A$\beta$ deposition correlates with downstream metabolic and cellular changes that promote tau seeding and aggregation. This synergy helps to explain why tau pathology emerges in close spatiotemporal association with cortical amyloid burden \citep{bielCombiningTauPETFMRI2022}.\\

\textbf{Predictive accuracy of models across cognitive stages is unclear.} a pivotal question in these modeling efforts is whether to employ template (group-averaged) connectomes or patient-specific connectomes \citep{thompsonPredictingSpreadAlzheimers, vasaNullModelsNetwork2022}. Although group-level templates can simplify modeling and reduce noise, each individual’s connectome exhibits unique topological features that might shape disease spread. Meanwhile, most existing approaches apply a single model across heterogeneous cohorts or focus only on one disease stage. Consequently, it remains unclear whether predictive accuracy differ between early asymptomatic or mild cognitive impairment (MCI) stages versus frank AD dementia \citep{heCoupledmechanismsModellingFramework2023}. Moreover, standard network diffusion models often ignore local factors, such as amyloid load, that can modulate tau progression.\\

To address these gaps, we systematically compare three models for predicting tau spread across the Alzheimer’s disease spectrum. Specifically, we evaluate the performance of Network Diffusion Model (NDM), Fisher--Kolmogorov--Petrovsky--Piscounov (FKPP), and a weighted FKPP model that incorporates regional A$\beta$ burden in connectomes derived from participants across four cognitive stages: cognitively normal (CN), early MCI (EMCI), late MCI (LMCI), and probable AD (AD). By examining how each model performs at each disease stage, we aim to provide insights into the mechanisms and predictive power of network-based disease spread models, as well as clarify the role of amyloid-tau interactions in shaping the topography of tau deposition \citep{thompsonDemonstrationOpensourceToolbox2024}. In doing so, we seek a more comprehensive perspective on AD progression across the clinical continuum and a foundation for future personalized modeling efforts.



